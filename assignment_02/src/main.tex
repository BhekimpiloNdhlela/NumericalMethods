\documentclass{article}
\usepackage[utf8]{inputenc}
\usepackage{physics}
\usepackage{listings}

\title{Applied Mathematics Assignment 02}
\author{Bhekimpilo Ndhlela (18998712)}

\date{02 March 2018}

\begin{document}

\maketitle
\pagebreak
\section*{Question 1}
\subsection*{Python Source Code: }
\begin{lstlisting}[language=Python]
#!/usr/bin/python
def fpi(f, x, k):
    for i in xrange(k):
        print i + 1, "\t{:.15f}".format(x)
        x = f(x)
    print "\n"

if __name__ == "__main__":
    x, k = 1, 10
    f = lambda x : (0.5 * x) + (1 / x)
    g = lambda y : ((2. * x) / 3.) + (2. / (3. * x))
    h = lambda z : (0.75 * x) + (0.5 * x)
    print "The itterations for F(x):"
    fpi(f, x, k)
    print "The itterations for G(x):"
    fpi(g, x, k)
    print "The itterations for H(x):"
    fpi(h, x, k)

\end{lstlisting}


\begin{center}
    \begin{tabular}{||c c c c||} 
    \hline
    i & a.) F(x) & b.) G(x) & c.) H(x) \\ [0.5ex] 
    \hline\hline
    1 & 1.000000000000000 & 1.000000000000000 & 1.000000000000000  \\ [1ex] 
    \hline
    2 & 1.500000000000000 & 1.333333333333333 & 1.250000000000000 \\ [1ex] 
    \hline
    3 & 1.416666666666667 & 1.333333333333333 & 1.250000000000000 \\ [1ex] 
    \hline
    4 & 1.414215686274510 & 1.333333333333333 & 1.250000000000000 \\ [1ex] 
    \hline
    5 & 1.414213562374690 & 1.333333333333333 & 1.250000000000000 \\ [1ex] 
    \hline
    6 & 1.414213562373095 & 1.333333333333333 & 1.250000000000000 \\ [1ex] 
    \hline
    7 & 1.414213562373095 & 1.333333333333333 & 1.250000000000000 \\ [1ex] 
    \hline
    8 & 1.414213562373095 & 1.333333333333333 & 1.250000000000000 \\ [1ex] 
    \hline
    9 & 1.414213562373095 & 1.333333333333333 & 1.250000000000000 \\ [1ex] 
    \hline
    10 & 1.414213562373095 & 1.333333333333333 & 1.250000000000000 \\ [1ex] 
    \hline
    \end{tabular}
\end{center}

\pagebreak


%question 2 (a.)
\section*{Question 2}
\subsection*{a.)}
\subsection*{Python Source Code: }
\begin{lstlisting}[language=Python]
#!/usr/bin/python
def sign(x):
    if x < 0:	return -1
    elif x > 0: return 1
    else:		return 0

def bisect(f, a, b, tol):
    fa, fb = f(a), f(b)

    # assumming f(a)f(b)<0 is satisfied!
    while (b-a)/2. > tol:
        c  = (a + b) / 2.0
        fc = f(c)
        if fc == 0:                  # c is a solution, done
            return c
        elif sign(fc)*sign(fa) < 0:  # a and c make the new interval
            b, fb = c, fc
        else:
            a, fa = c, fc
    return (a+b)/2.                  # new midpoint is best estimate

if __name__ == "__main__":
    import sys
    from math import (cos, tan)
    f = lambda theta: 20.0 * tan(theta) - ((20.0**2 * 9.81) / (2 * (17.0**2) * cos(theta)**2)) - 3
    print "{:.15f}".format(bisect(f,0,1, 1.0e-5))
    
\end{lstlisting}

\begin{center}
  \textbf{ Answer: 0.554908752441406}
\end{center}


\pagebreak
\section*{Question 3}
\subsection*{Python Source Code for question 3 a.), b.) and c.): }
\begin{lstlisting}[language=Python]
def question_3a(f, debug=True):
    x = [.9, 0, 0, 0, 0, 0]
    for i in xrange(5):
        fx, dx = f(x[i])
        x[i+1] = x[i] - (fx / dx)

    if debug is True:
        print "Question 3 (a) results:"
        for i in xrange(len(x)):
            print i + 1, "\t", "{:.15f}".format(x[i])

def question_3b(g, debug=True):
    x = [.9, 0, 0, 0, 0, 0]
    for i in xrange(5):
        gx, dx = g(x[i])
        x[i+1] = x[i] - (gx / dx)

    if debug is True:
        print "Question 3 (b) results:"
        for i in xrange(0 ,len(x)):
            print i + 1, "\t", "{:.15f}".format(x[i])

def question_3d(g, debug=True):
    x = [.9, 0, 0, 0, 0, 0]
    # m == multiplicity of the function g(x)
    m = 2
    for i in xrange(5):
        gx, dx = g(x[i])
        x[i+1] = x[i] - (m * (gx / dx))

    if debug is True:
        print "Question 3 (d) results:"
        for i in xrange(0 ,len(x)):
            print i + 1, "\t", "{:.15f}".format(x[i])

from numpy import exp
f = lambda x: (x * exp(x - 1) - 1, exp(x-1) + x * exp(x-1))
g = lambda x: (-x * exp(1 - x) + 1, -exp(1-x) + x * exp(1-x))

question_3a(f)
question_3b(g)
question_3d(g)
    
\end{lstlisting}

\subsection*{a.)}

\textbf{$x_0 = 0.900000000000000$}
\begin{center}
    \begin{tabular}{||c c||} 
    \hline
    \textbf{i} & \textbf{Secant Method} \\ [0.5ex] 
    \hline\hline
    1 & 0.900000000000000 \\ [1ex] 
    \hline
    2 & 1.007984693724025 \\ [1ex] 
    \hline
    3 & 1.000047584190120 \\ [1ex] 
    \hline
    4 & 1.000000001698142 \\ [1ex] 
    \hline
    5 & 1.000000000000000 \\ [1ex] 
    \hline
    6 & 1.000000000000000 \\ [1ex] 
    \hline
    \end{tabular}
\end{center}


\subsection*{b.)}
\textbf{$x_0 = 0.900000000000000$}
\begin{center}
    \begin{tabular}{||c c||} 
    \hline
    \textbf{i} & \textbf{Secant Method} \\ [0.5ex] 
    \hline\hline
    1 & 0.900000000000000 \\ [1ex] 
    \hline
    2 & 0.948374180359597 \\ [1ex] 
    \hline
    3 & 0.973748560382854 \\ [1ex] 
    \hline
    4 & 0.986760173690201 \\ [1ex] 
    \hline
    5 & 0.993350967791509 \\ [1ex] 
    \hline
    6 & 0.996668127855918 \\ [1ex] 
    \hline
    \end{tabular}
\end{center}
\pagebreak
\subsection*{c.)}

\[f(x) = xe^{(x-1)} - 1\] \[ f(r) = f(1) = 0\]
\[f^`(x) = e^{(x-1)} + xe^{(x-1)}\] \[f^`(r) = f`(1) = 2 \ne 0\]
\[multiplicity = m = 1 \]\\ \\ \\

\[g(x) = -xe^{(1-x)} + 1\] \[ g(r) = g(1) = 0\]
\[g^`(x) = -e^{(1-x)} + xe^{(1-x)}\] \[ g`(r) = g`(1) = 0\]
\[g^``(x) = e^{(1-x)} -(x-1)e^{(1-x)} = -(x-2)e^{(1-x)}\] \[g``(r) = g``(1) = 1 \ne 0\]
\[multiplicity = m = 2\]


\textbf{Since: $m = 1$, for $f(x)$, then $f(x)$ is well conditioned.\\ However, $m = 2$ for $g(x)$, this implies that $g(x)$ is ill-conditioned.}

\subsection*{d.)}
\textbf{$x_0 = 0.900000000000000$ \\ $multiplicity = m = 2$}
\begin{center}
    \begin{tabular}{||c c||} 
    \hline
    \textbf{i} & \textbf{Secant Method} \\ [0.5ex] 
    \hline\hline
    1 & 0.900000000000000 \\ [1ex] 
    \hline
    2 & 0.996748360719194 \\ [1ex] 
    \hline
    3 & 0.999996478477208 \\ [1ex] 
    \hline
    4 & 1.000000000070078 \\ [1ex] 
    \hline
    5 & 1.000000000070078 \\ [1ex] 
    \hline
    6 & 1.000000000070078 \\ [1ex] 
    \hline
    \end{tabular}
\end{center}
\textbf{Since: $m = 2$ for $g(x)$, this implies that $g(x)$ is ill-conditioned, and hence the root has not been computed to full precision with this method (This is because of rounding off errors). The ill-conditioning has not been thwarted.}



\section*{Question 4}
\subsection*{Python Source Code: }
\begin{lstlisting}[language=Python]
def secant_method(x, debug=True):
    appr_list = [1., 0.5, 0., 0., 0., 0., 0., 0., 0.]
    for i in xrange(1, len(appr_list) - 1):
        num = appr_list[i] * appr_list[i - 1] + x
        den = appr_list[i] + appr_list[i - 1]
        appr_list[i + 1] = num / den

    if debug is True:
        print "The Secant Method, DEBUG_MODE: ON:"
        for i in xrange(len(appr_list)):
            print i + 1, "\t", "{0:.15f}".format(appr_list[i])
    return appr_list

def newton_method(x, debug=True):
    appr_list = [1., 0., 0., 0., 0., 0., 0., 0., 0.]
    for i in xrange(len(appr_list) - 1):
        num = appr_list[i]**2 + x
        den = appr_list[i] * 2
        appr_list[i + 1] = num / den

    if debug is True:
        print "The Newton Method, DEBUG_MODE: ON:"
        for i in xrange(len(appr_list)):
            print i + 1, "\t", "{0:.15f}".format(appr_list[i])
    return appr_list
import math
new_res, sec_res = newton_method(1. / 9.), secant_method(1. / 9.)
\end{lstlisting}

\begin{center}
    \begin{tabular}{||c c c||} 
    \hline
    \textbf{i} & \textbf{Secant Method} & \textbf{Newton's Method}\\ [0.5ex] 
    \hline\hline
    1 & 1.000000000000000 & 1.000000000000000  \\ [1ex] 
    \hline
    2 & 0.500000000000000 & 0.555555555555556 \\ [1ex] 
    \hline
    3 & 0.407407407407407 & 0.377777777777778 \\ [1ex] 
    \hline
    4 & 0.346938775510204 & 0.335947712418301 \\ [1ex] 
    \hline
    5 & 0.334669338677355 & 0.333343506014598 \\ [1ex] 
    \hline
    6 & 0.333360001066709 & 0.333333333488554 \\ [1ex] 
    \hline
    7 & 0.333333386666671 & 0.333333333333333 \\ [1ex] 
    \hline
    8 & 0.333333333335467 & 0.333333333333333 \\ [1ex] 
    \hline
    9 & 0.333333333333333 & 0.333333333333333 \\ [1ex] 
    \hline
    \end{tabular}
\end{center}




\section*{Question 5}

\subsection*{Python Source Code: }
\begin{lstlisting}[language=Python]
#!/usr/bin/python
def newton_method_sys(fxy, j0, j1, debug=True):
    xn = zeros((2, 9))      #store itteration results for x^[n + 1]
    jx = zeros((2, 2))      #store currant itteration jacobian inverse
    fx = zeros((2, 1))      #store the results of the f(x^[n]) for a particular itteration
    sx = zeros((2, 1))

    for i in xrange(len(xn[1]) - 1):
        jx[0][0], jx[0][1] = j0(xn[0][i], xn[1][i])
        jx[1][0], jx[1][1] = j1(xn[0][i], xn[1][i])
        fx[0][0], fx[1][0] = fxy(xn[0][i], xn[1][i])
        sx = linalg.solve(negative(jx), fx)
        xn[0][i + 1] = xn[0][i] + sx[0]
        xn[1][i + 1] = xn[1][i] + sx[1]

    if debug is True:
        print "xn = |", xn[0][-1], xn[1][-1], "|"

if __name__ == "__main__":
    from numpy import (array, zeros, exp, linalg, negative)
    from math import (cos, sin)
    fxy = lambda x, y: (x * exp(y) + y - 7, sin(x) - cos(y))
    j0 = lambda x, y: (exp(y), x * exp(y) + 1)
    j1 = lambda x, y: (cos(x), sin(y))
    newton_method_sys(fxy, j0, j1)
else:
    import sys
    sys.exit("please run as client...")

\end{lstlisting}
  \textbf{when:\\}
  \textbf{$x_0 = [0, 0]^T$ \\ \textbf{Then:}\\  $x^* = 0.0199681607333$\\ \textbf{And also:}\\ $f(x^*) = 4.73235714112$}

\end{document}


